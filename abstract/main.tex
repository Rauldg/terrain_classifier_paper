\documentclass{article}

% Language setting
% Replace `english' with e.g. `spanish' to change the document language
\usepackage[english]{babel}

% Set page size and margins
% Replace `letterpaper' with `a4paper' for UK/EU standard size
\usepackage[letterpaper,top=2cm,bottom=2cm,left=3cm,right=3cm,marginparwidth=1.75cm]{geometry}

% Useful packages
\usepackage{amsmath}
\usepackage{graphicx}
\usepackage{todonotes}
\usepackage[colorlinks=true, allcolors=blue]{hyperref}

\title{Rock Implementation of an On-Board Terrain Classifier based on Proprioceptive Sensor Data for a Planetary Rover}
\author{
Dominguez. Raul \and Kuhr, Lennart \and Babel, Jonathan
}

\begin{document}
\maketitle

%\begin{abstract}

%\end{abstract}

\section{Introduction}

\todo[inline]{Contact Giulio Reina in case he and someone else from his team wants to be included as authors}


Automated terrain awareness, the correct modeling of the surfaces transited by a rover and its’ classification, is a key factor for successful navigation. 
\todo[inline]{more on motivation why is this important}
In this publication a software component that is capable of classifying the traversed terrain type based on proprioceptive sensor data is presented. 
The component uses a Support Vector Machine (SVM) algorithm \cite{vapnik1992,cristianini2000} in its core and it is integrated into the software control architecture of the hybrid locomotion rover SherpaTT \cite{cordes2018}, such that it can be executed sufficiently fast during navigation. 
The functionality has been integrated using the framework Robotic Construction Toolkit (Rock).
One of the main challenges addressed by the middleware component is to generate online from the multifrequency datastreams, matrices of synchronous sample values for classification.

%During the traverse the terrain classifier component uses proprioceptive sensor data -force torque sensors and joints- and dataproducts -acceleration estimates- to classify between the three terrain types.
During the rover's navigation the terrain classifier component uses force torque sensors, joints data and body acceleration estimates to classify the surface into one of three terrain types.
The three terrain types -\emph{sand, compact sand and concrete}- represent distinct classes of surfaces characterized by its deformability and fiction properties but are hard to identify with exteroceptive sensors since they are visually similar.

In order to allow other onboard components to utilize the classification results (e.g. to improve the navigation), the results need to be available at a fast enough pace so that the classification results do correspond to the currently traversed surface while avoiding the loss of sensor data messages due to full queues.

% The SVM model training software is implemented to enable the use of a variable constellation of logged datasets. Moreover, it can be used to compare both the offline and online classification performance. 
For the analysis of the classification performance onboard SherpaTT, the results generated during several test traverses were logged and compared with the ones from the training sets.
Several tests were conducted which validated that the developed code can be successfully executed on SherpaTT.
Moreover, the classification performance achieved onboard SherpaTT was identified during testruns on terrain that represent at least one of the three terrain type classes.

%\begin{figure}
%\centering
%\includegraphics[width=0.3\textwidth]{frog.jpg}
%\caption{\label{fig:frog}This frog was uploaded via the file-tree menu.}
%\end{figure}
\nocite{*}
\bibliographystyle{alpha}
\bibliography{sample}

\end{document}