
\section{Implementation}

The classifier control loop has a target frequency of 100 $Hzs$, matching the highest input frequency (joint status). 
To address the issue of other streams having lower frequency, the simple solution of using the same data as input for several entries in the classifier input matrices was taken. 
The classification is performed on matrices composed from information generated from 100 samples. 
Thus a classification is generated every second.
In other words, every 100 samples, a matrix input for the classifier is completed and a new classification is done, this happens every 1 $s$.
Since the speed of the rover was 0.1 $m/s$, the resolution of the categorized patches of terrain is 10 $cm$ long.  


The classification results need to be available at a fast enough pace to allow other onboard components take advantage of the results (e.g. to improve navigation) and to ensure that the classification results correspond to the currently traversed surface. 
Likewise, the loss of data samples due to full queues on the input of the processing components must be avoided. 

The diagram in Figure~\ref{fig:overview} presents the implementation approach of the terrain classifier in Rock. 
The first step which needs to be done efficiently is the extraction from the data sample of the relevant fields of data which is done using the \emph{type to vector} Rock tool.
At each execution of the \emph{updateHook} (every 0.01) new data is stacked in the input matrix.
Every 100 \emph{updateHooks} an input matrix is finished and the component triggers the computation of the physical and statisticall features and finally the classification, also named prediction.
The output of the classification as well as the of the feature computation are delivered to the calling component which retrives the values through its output ports. 

\begin{figure}[!htbp]
    \centering
    \includegraphics[width=0.9\textwidth]{../figures/OverviewTC2.pdf}
    \caption{\label{fig:overview}Overview of the terrain classifier library and Rock integration.}
\end{figure}

\subsection{Training Data}

An essential aspect needed to achieve a good classification performance is to have a consistent and large data collection available for training and testing. 
The data collection available for this implementation consists of a dataset composed of traverses that were acquired by remote commanding the rover over the 3 examined terrain types: \emph{loose sand}, \emph{compact sand} and \emph{concrete}. 
One traverse in the dataset corresponds to the forward and backward driving of 10 $m$. 
The following conditions were the same during all traverses: (1) Fixed wheel configuration, (2) Surfaces without inclination, (3) Traverse speed 0.1 $m/s$, (4) Straight traverses and (5) Electric power generator is running, which causes vibrations.
Figure~\ref{fig:TestLocs} shows images of the rover traversing the different terrains during the data collection.


\begin{figure}[!htb]
   \centering
    \subcaptionbox
        {Loose Sand}
        {\includegraphics[width=0.32\textwidth]{../figures/unprepsand.png}}
    \subcaptionbox
        {Compact Sand}
        {\includegraphics[width=0.32\textwidth]{../figures/compact.png}}
    \subcaptionbox
        {Concrete}
        {\includegraphics[width=0.32\textwidth]{../figures/concrete_v2.png}}
    \caption{The test locations where the data sets were acquired.}
    \label{fig:TestLocs}
\end{figure}



In terms of quantity, the data collection consists of 2200 training samples for each of the 76 features. 
Regarding data balance, the data gathered from the terrain type loose sand represent 28\% of the total and the concrete and compact sand represent 36\% each. 
When generating a SVM model, the available data collection is divided into training and testing sets. 
For the presented classifier one of the traverses of each terrain is used for testing and all other traverses are used for training. 
This gives a ratio of about 25\%/75\%.
